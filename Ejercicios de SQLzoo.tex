\documentclass[10pt, oneside]{article} 
\usepackage{amsmath, amsthm, amssymb, calrsfs, wasysym, verbatim, bbm, color, graphics, geometry}
\usepackage{hyperref}
\geometry{tmargin=.75in, bmargin=.75in, lmargin=.75in, rmargin = .75in}  

\newcommand{\R}{\mathbb{R}}
\newcommand{\C}{\mathbb{C}}
\newcommand{\Z}{\mathbb{Z}}
\newcommand{\N}{\mathbb{N}}
\newcommand{\Q}{\mathbb{Q}}
\newcommand{\Cdot}{\boldsymbol{\cdot}}

\newtheorem{thm}{Theorem}
\newtheorem{defn}{Definition}
\newtheorem{conv}{Convention}
\newtheorem{rem}{Remark}
\newtheorem{lem}{Lemma}
\newtheorem{cor}{Corollary}


\title{SQLzoo\\ Solved problems}
\author{Alejandro Florido Reyes}
\date{Academic Year 2019-2020}

\begin{document}

\maketitle
\tableofcontents

\vspace{.25in}

\section{SELECT basics}
\begin{itemize}
	\item Introducing the world table of countries
	
	\begin{center}
		\begin{minipage}{.4\linewidth}
			SELECT population FROM world\\
			WHERE name = 'Germany
		\end{minipage}
	\end{center}

	\item Scandinavia
	
	\begin{center}
		\begin{minipage}{.6\linewidth}
			SELECT name, population FROM world\\
			WHERE name IN ('Sweden', 'Norway', 'Denmark')
		\end{minipage}
	\end{center}
	
	\item Just the right size
	
	\begin{center}
		\begin{minipage}{.6\linewidth}
			SELECT name, area FROM world \\
			WHERE area BETWEEN 200000 AND 250000
		\end{minipage}
	\end{center}
	
\end{itemize}

\section{SELECT names}
\begin{itemize}
	\item Pattern matching strings
	1.
	\begin{center}
		\begin{minipage}{.6\linewidth}
			SELECT name FROM world \\
			WHERE name LIKE 'Y\%'
		\end{minipage}
	\end{center}
	2.
	\begin{center}
		\begin{minipage}{.6\linewidth}
			SELECT name FROM world\\
			WHERE name LIKE '\%y'
		\end{minipage}
	\end{center}
	3.
	\begin{center}
		\begin{minipage}{.6\linewidth}
			SELECT name FROM world \\
			WHERE name LIKE '\%x\%'
		\end{minipage}
	\end{center}
	4.
	\begin{center}
		\begin{minipage}{.6\linewidth}
			SELECT name FROM world\\
			WHERE name LIKE '\%land'
		\end{minipage}
	\end{center}
	5.
	\begin{center}
		\begin{minipage}{.6\linewidth}
			SELECT name FROM world \\
			WHERE name LIKE 'C\%ia'
		\end{minipage}
	\end{center}
	6.
	\begin{center}
		\begin{minipage}{.6\linewidth}
			SELECT name FROM world \\
			WHERE name LIKE '\%oo\%'
		\end{minipage}
	\end{center}
	7.
	\begin{center}
		\begin{minipage}{.6\linewidth}
			SELECT name FROM world \\
			WHERE name LIKE '\%a\%a\%a\%'
		\end{minipage}
	\end{center}
	8.
	\begin{center}
		\begin{minipage}{.6\linewidth}
			SELECT name FROM world \\
			WHERE name LIKE '\_t\%'
		\end{minipage}
	\end{center}
	9.
	\begin{center}
		\begin{minipage}{.6\linewidth}
			SELECT name FROM world \\
			WHERE name LIKE '\%o\_\_o\%'
		\end{minipage}
	\end{center}
	10.
	\begin{center}
		\begin{minipage}{.6\linewidth}
			SELECT name FROM world \\
			WHERE length(name)=4
		\end{minipage}
	\end{center}
	11.
	\begin{center}
		\begin{minipage}{.6\linewidth}
			SELECT name FROM world \\
			WHERE name=capital
		\end{minipage}
	\end{center}
	12.
	\begin{center}
		\begin{minipage}{.6\linewidth}
			SELECT name FROM world \\
			WHERE capital = CONCAT(name,' city')
		\end{minipage}
	\end{center}
	13.
	\begin{center}
		\begin{minipage}{.6\linewidth}
		SELECT capital, name FROM world \\
		WHERE capital LIKE CONCAT('\%',name,'\%')	
		\end{minipage}
	\end{center}
	14.
	\begin{center}
		\begin{minipage}{.6\linewidth}
			SELECT capital, name FROM world \\
			WHERE capital LIKE CONCAT('\%',name,'\%') \\
			AND length(capital)$>$length(name)
		\end{minipage}
	\end{center}
	15.
	\begin{center}
		\begin{minipage}{.6\linewidth}
			\color{red}SELECT name, REPLACE(capital,name,'') AS ext FROM world \\
			WHERE capital LIKE CONCAT('\%',name,'\%') AND \\
			length(capital)$>$length(name)
		\end{minipage}
	\end{center}

	\href{https://sqlzoo.net/wiki/REPLACE}{Use of the function REPLACE( , , )}
	
\end{itemize}

\section{SELECT from World}

\subsection{Introduction}

SELECT name, continent, population FROM world

\subsection{Large countries}

SELECT name FROM world WHERE population >= 200000000

\subsection{Per capita GDP}

SELECT name, gdp/population FROM world WHERE population >= 200000000

\subsection{South America in millions}

SELECT name, population/1000000 FROM world WHERE continent = 'South America'

\subsection{France, Germany, Italy}

SELECT name, population FROM world WHERE name IN ('France', 'Germany', 'Italy')

\subsection{United}

SELECT name FROM world WHERE name LIKE '\%United\%'

\subsection{Two ways to be big}

SELECT name, population, area FROM world WHERE population >250000000 OR area>3000000

\subsection{One or the others (but not both)}

SELECT name, population, area FROM world WHERE (population >250000000 AND area<=3000000) OR (area>3000000 AND population<=250000000)

\subsection{Rounding}

SELECT name, ROUND(population/1000000,2), ROUND(gdp/1000000000,2) FROM world WHERE continent='South America'

\subsection{Trillion dollar economies}

SELECT name, ROUND(gdp/population,\color{red}-3\color{black}) FROM world WHERE gdp>=1000000000000

\subsection{Name and capital have the same length}

SELECT name, capital FROM world WHERE length(name)=length(capital)

\subsection{Matching name and capital}

\color{red}SELECT name, capital FROM world 
WHERE LEFT(name,1) = LEFT(capital,1)
AND name<>capital

\color{black}\href{https://sqlzoo.net/wiki/LEFT}{Use of the function LEFT(string,number)}

\subsection{All the vowels}

SELECT name FROM world WHERE name LIKE '\%a\%' AND name LIKE '\%e\%' AND name LIKE '\%i\%' AND name LIKE '\%o\%' AND name LIKE '\%u\%' AND name NOT LIKE '\% \%'

\section{SQL from nobel}

\subsection{Winners from 1950}

SELECT yr, subject, winner \\
FROM nobel \\
WHERE yr=1950 \\

\subsection{1962 Literature}

SELECT winner \\
FROM nobel \\
WHERE yr=1962 AND subject = 'Literature' \\

\subsection{Albert Einstein}

SELECT yr, subject \\
FROM nobel \\
WHERE winner = 'Albert Einstein' \\

\subsection{Recent Peace Prizes}

SELECT winner \\
FROM nobel \\
WHERE yr $>=$2000 AND subject = 'Peace' \\

\subsection{Literature in the 1980's}

SELECT yr, subject, winner\\
FROM nobel\\
WHERE (yr BETWEEN 1980 AND 1989) AND subject = 'Literature'\\

\subsection{Only presidents}

SELECT * \\
FROM nobel\\
WHERE winner IN ('Barack Obama', 'Jimmy Carter', 'Woodrow Wilson', 'Theodore Roosevelt')\\

\subsection{John}

SELECT winner\\
FROM nobel\\
WHERE winner LIKE 'John \%'\\

\subsection{Chemistry and Physics from different years}

SELECT *\\
FROM nobel\\
WHERE (subject = 'Physics' AND yr=1980) OR (subject = 'Chemistry' AND yr=1984)\\

\subsection{Exclude Chemists and Medics}

SELECT *
FROM nobel
WHERE yr=1980 AND subject NOT IN ('Medicine', 'Chemistry')

\subsection{Early Medicine, Late Literature}

SELECT *\\
FROM nobel\\
WHERE (yr$<$1910 AND subject = 'Medicine') OR (yr$>=$2004 AND subject = 'Literature')\\

\subsection{Umlaut}

SELECT *\\
FROM nobel\\
WHERE winner = 'Peter \color{red}Grünberg\color{black}'\\

\subsection{Apostrophe}

SELECT *\\
FROM nobel\\
WHERE winner = 'Eugene O\color{red}''\color{black}NEILL'\\

\color{blue}*Note. You can't put a single quote in a quote string directly. You can use two single quotes within a quoted string.\color{black}

\subsection{Knights of the realm}

SELECT winner, yr, subject\\
FROM nobel\\
WHERE winner LIKE 'Sir \%'\\
ORDER BY yr DESC, winner\\

\subsection{Chemistry and Physics last}
\color{red}
SELECT winner, subject \\
FROM nobel\\
WHERE yr=1984\\
ORDER BY (subject IN ('Chemistry', 'Physics')) ASC, subject, winner\\

\color{blue}
*Note. subject IN ('Chemistry', 'Physics') es una condición booleana si se aplica, por ejemplo, en el SELECT (devuelve 0 o 1 para cada fila en una nueva columna). \color{black}

\section{SELECT within SELECT}

\subsection{Bigger than Russia}

SELECT name\\
FROM world\\
WHERE population > (\color{red}SELECT population FROM world WHERE name = 'Russia'\color{black})\\

\subsection{Richer than UK}

SELECT name\\
FROM world\\
WHERE continent = 'Europe' AND gdp/population > (SELECT gdp/population FROM world WHERE name = 'United Kingdom')\\

\subsection{Neighbours of Argentina and Australia}

SELECT name, continent
FROM world
WHERE continent IN (\color{red}(SELECT continent FROM world WHERE name='Argentina')\color{black},\color{red}(SELECT continent FROM world WHERE name = 'Australia')\color{black})
ORDER BY name

\subsection{Between Canada and Poland}

SELECT name,population\\
FROM world\\
WHERE population $>$ (SELECT population FROM world WHERE name ='Canada') AND population $<$ (SELECT population FROM world WHERE name = 'Poland')\\

\subsection{Percentages of Germany}

SELECT name AS Name, CONCAT(ROUND(population/(SELECT population FROM world WHERE name = 'Germany')*100,0), '\%') AS Percentage\\
FROM world\\
WHERE continent = 'Europe'\\

\color{blue} El comando ALL. \color{black}Se puede usar junto con los operadores de comparación ($>$, $<$, etc.) y seguido de una subquery para indicar "el mayor/menor/..." de todos los valores devueltos por la subquery.

\subsection{Bigger than every country in Europe}

SELECT name\\
FROM world\\
WHERE gdp $>$ ALL(SELECT gdp FROM world WHERE continent = 'Europe' and gdp IS NOT NULL)\\

or...\\

SELECT name\\
FROM world\\
WHERE gdp > (SELECT max(gdp) FROM world WHERE continent = 'Europe')\\

\subsection{Largest in each continent}

SELECT w1.continent, w1.name, w1.area\\
FROM world AS w1\\
WHERE w1.area = (SELECT max(w2.area) FROM world AS w2 WHERE w1.continent=w2.continent)\\

\subsection{First country of each continent (alphabetically)}

SELECT w1.continent, w1.name\\
FROM world AS w1\\
\color{red}WHERE w1.name $<=$ ALL(SELECT w2.name FROM world AS w2 WHERE w1.continent = w2.continent)\\\color{blue}

*Nota. \color{black}Si se utilizan los operadores de comparación con strings, el "más pequeño" sería el primero por orden alfabético.

\subsection{Difficult Questions That Utilize Techniques Not Covered In Prior Sections}

\subsubsection{Find the continents where all countries have a population <= 25000000. Then find the names of the countries associated with these continents. Show name, continent and population.}

SELECT name, continent, population\\
FROM world AS w1\\
WHERE 25000000 $>$= (SELECT max(population) FROM world AS w2 WHERE w1.continent = w2.continent)\\

or...\\

SELECT w1.name, w1.continent, w1.population\\
FROM world AS w1\\
WHERE 25000000 $>$= ALL(SELECT w2.population FROM world AS w2 WHERE w1.continent = w2.continent)\\

\subsubsection{\color{red}Some countries have populations more than three times that of any of their neighbours (in the same continent). Give the countries and continents.}

SELECT w1.name, w1.continent\\
FROM world AS w1\\
WHERE w1.population $>$ 3*(SELECT max(w2.population) FROM world AS w2 WHERE w1.continent = w2.continent AND w1.name != w2.name)\\

\color{blue}*Note. \color{black}En este caso no se puede usar ALL, ya que se daría  una contradicción entre dicho comando y la cláusula w1.name != w2.name.

\section{SUM and COUNT}

\subsection{Total world population}

\subsubsection{Show the total population of the world.}

SELECT SUM(population) \\
FROM world\\

\subsection{List of continents}

\subsubsection{List all the continents - just once each.}

SELECT DISTINCT continent\\
FROM world\\

\subsection{GDP of Africa}

\subsubsection{Give the total GDP of Africa}

SELECT SUM(gdp)\\
FROM world\\
WHERE continent = 'Africa'\\

\subsection{Count de big countries}

\subsubsection{How many countries have an area of at least 1000000}

SELECT COUNT(*)\\
FROM world\\
WHERE area >= 1000000\\

\subsubsection{Baltic states population}

\subsection{What is the total population of ('Estonia', 'Latvia', 'Lithuania')}

SELECT SUM(population)\\
FROM world\\
WHERE name IN ('Estonia','Latvia','Lithuania')\\

\subsection{Counting the countries of each continent}

\subsubsection{For each continent show the continent and number of countries.}

SELECT continent, COUNT(*)\\
FROM world\\
GROUP BY continent\\

\subsection{Counting big countries in each continent}

\subsubsection{For each continent show the continent and number of countries with populations of at least 10 million.}

SELECT continent, count(*)\\
FROM world\\
WHERE population $>$ 10000000\\
GROUP BY continent\\

\subsection{Counting big continents}

\subsubsection{List the continents that have a total population of at least 100 million.}

SELECT continent\\
FROM world\\
GROUP BY continent\\
HAVING SUM(population)>=100000000\\

\section{JOIN}

\subsection{JOIN and UEFA EURO 2012}

\subsubsection{The first example shows the goal scored by a player with the last name 'Bender'. The * says to list all the columns in the table - a shorter way of saying matchid, teamid, player, gtime. Modify it to show the matchid and player name for all goals scored by Germany. To identify German players, check for: teamid = 'GER'}

SELECT matchid, player \\
FROM goal\\
WHERE teamid = 'GER'\\

\subsubsection{From the previous query you can see that Lars Bender's scored a goal in game 1012. Now we want to know what teams were playing in that match. Notice in the that the column matchid in the goal table corresponds to the id column in the game table. We can look up information about game 1012 by finding that row in the game table. Show id, stadium, team1, team2 for just game 1012}

SELECT id, stadium, team1, team2\\
FROM game INNER JOIN goal ON game.id = goal.matchid\\
WHERE matchid = 1012 AND player = 'Lars Bender'\\

\subsubsection{The code below shows the player, teamid, stadium and mdate for every German goal.}

SELECT player, teamid, stadium, mdate\\
FROM game JOIN goal ON goal.matchid = game.id\\
WHERE teamid = 'GER'\\

\subsubsection{Show the team1, team2 and player for every goal scored by a player called Mario.}

SELECT team1, team2, player \\
FROM game INNER JOIN goal ON goal.matchid = game.id\\
WHERE player LIKE 'Mario \%'\\

\subsubsection{The table eteam gives details of every national team including the coach. Show player, teamid, coach, gtime for all goals scored in the first 10 minutes gtime$<=$10}

SELECT player, teamid, coach, gtime \\
FROM goal INNER JOIN eteam ON eteam.id = goal.teamid\\
WHERE gtime $<=$ 10\\

\subsubsection{To JOIN game with eteam you could use either game JOIN eteam ON (team1=eteam.id) or game JOIN eteam ON (team2=eteam.id). Notice that because id is a column name in both game and eteam you must specify eteam.id instead of just id. List the dates of the matches and the name of the team in which 'Fernando Santos' was the team1 coach.}

SELECT mdate, teamname\\
FROM game INNER JOIN eteam ON eteam.id = game.team1\\
WHERE coach = 'Fernando Santos'\\

\subsubsection{List the player for every goal scored in a game where the stadium was 'National Stadium, Warsaw'}

SELECT player\\
FROM game INNER JOIN goal ON game.id = goal.matchid\\
WHERE stadium = 'National Stadium, Warsaw'\\

\subsection{More difficult questions}

\subsubsection{Show the name of all players who scored a goal against Germany}

SELECT DISTINCT player \\
FROM goal JOIN game ON game.id = goal.matchid \\
WHERE (game.team1 = 'GER' OR game.team2 = 'GER') AND goal.teamid != 'GER' \\

\subsubsection{Show teamname and the total number of goals scored}

SELECT teamname, COUNT(goal.teamid)\\
FROM goal JOIN eteam ON eteam.id = goal.teamid\\
GROUP BY teamname\\

\subsubsection{Show the stadium and the number of goals scored in each stadium.}

SELECT stadium, COUNT(teamid)\\
FROM game INNER JOIN goal ON game.id = goal.matchid\\
GROUP BY stadium\\

\subsubsection{For every match involving 'POL', show the matchid, date and the number of goals scored}

SELECT matchid, mdate, COUNT(goal.matchid)\\
FROM game INNER JOIN goal ON game.id = goal.matchid\\
WHERE team1 = 'POL' OR team2 = 'POL'\\
GROUP BY matchid, mdate\\

\subsubsection{For every match where 'GER' scored, show matchid, match date and the number of goals scored by 'GER'}

SELECT matchid, mdate, COUNT(goal.matchid)\\
FROM game INNER JOIN goal ON goal.matchid = game.id\\
WHERE (team1 = 'GER' OR team2 = 'GER') AND goal.teamid = 'GER'\\
GROUP BY goal.matchid, game.mdate \\

\subsubsection{List every match with the goals scored by each team as shown. This will use "CASE WHEN" which has not been explained in any previous exercises.}

SELECT mdate, team1, \color{red}SUM(CASE\\
WHEN game.team1 = goal.teamid THEN 1\\
ELSE 0\\
END)\color{black}\\
AS score1,\\
team2, \color{red}SUM(CASE\\
WHEN game.team2=goal.teamid THEN 1\\
ELSE 0\\
END)\color{black}\\
AS score2\\
FROM game \color{red}LEFT JOIN \color{black}goal ON game.id = goal.matchid\\
GROUP BY mdate, team1, team2\\
ORDER BY mdate, matchid, team1, team2\\

Hay que usar el LEFT para incluir partidos que quedaron empatados a 0 puntos. No se puede usar el COUNT dentro del CASE y ahorrarnos el SUM porque entonces los resultados se separan de tal forma que si ambos equipos marcan gol en el mismo partido, se obtienen dos filas en lugar de una.\\

Link para revisar el uso de CASE: \hyperref{https://sqlzoo.net/wiki/CASE}{}{}{https://sqlzoo.net/wiki/CASE}.

\section{More JOIN operations}

\subsection{1962 movies}

SELECT id, title\\
FROM movie\\
WHERE yr = 1962\\

\subsection{When was Citizen Kane released?}

SELECT yr\\
FROM movie\\
WHERE title = 'Citizen Kane'\\

\subsection{Star Trek movies}

SELECT id, title, yr \\
FROM movie \\
WHERE title LIKE '\%Star Trek\%'\\
ORDER BY yr\\

\subsection{id for actor Glenn Close}

What id number does the actor 'Glenn Close' have?\\ 

SELECT id \\
FROM actor\\
WHERE name = 'Glenn Close'\\

\subsection{id for Casablanca}

What is the id of the film 'Casablanca'?\\

SELECT id\\
FROM movie\\
WHERE title = 'Casablanca'\\

\subsection{Cast list for Casablanca}

Obtain the cast list for 'Casablanca'.\\

what is a cast list?\\
The cast list is the names of the actors who were in the movie.\\

Use movieid=11768, (or whatever value you got from the previous question)\\

SELECT actor.name\\
FROM (actor INNER JOIN casting ON actor.id = casting.actorid) \\
INNER JOIN movie ON movie.id = casting.movieid\\
WHERE title = 'Casablanca'\\

\subsection{Alien cast list}

Obtain the cast list for the film 'Alien'\\

SELECT actor.name\\
FROM (actor INNER JOIN casting ON actor.id = casting.actorid)\\
INNER JOIN movie ON movie.id = casting.movieid\\
WHERE title = 'Alien'\\

*Nota. Sí, es exactamente igual que el anterior.

\subsection{Harrison Ford movies}

List the films in which 'Harrison Ford' has appeared.\\

SELECT title 
FROM (movie INNER JOIN casting ON movie.id = casting.movieid) 
INNER JOIN actor ON actor.id = casting.actorid 
WHERE name = 'Harrison Ford'

\subsection{Harrison Ford as a supporting actor}

List the films where 'Harrison Ford' has appeared - but not in the starring role. [Note: the ord field of casting gives the position of the actor. If ord=1 then this actor is in the starring role]\\

SELECT title\\
FROM (movie INNER JOIN casting ON movie.id = casting.movieid)\\ 
INNER JOIN actor ON actor.id = casting.actorid\\
WHERE ord != 1 AND name = 'Harrison Ford'\\

\subsection{Lead actors in 1962 movies}

List the films together with the leading star for all 1962 films.\\

SELECT title, name\\
FROM (movie INNER JOIN casting ON movie.id = casting.movieid) \\
INNER JOIN actor ON actor.id = casting.actorid\\
WHERE yr = '1962' AND ord = 1\\

\subsection{Busy years for Rock Hudson}

Which were the busiest years for 'Rock Hudson', show the year and the number of movies he made each year for any year in which he made more than 2 movies.\\

SELECT yr, COUNT(title)\\
FROM (movie INNER JOIN casting ON movie.id = casting.movieid)\\ 
INNER JOIN actor ON actor.id = casting.actorid\\
WHERE name = 'Rock Hudson'\\
GROUP BY yr\\
HAVING COUNT(title) > 2\\

\subsection{Lead actor in Julie Andrews movies}

List the film title and the leading actor for all of the films 'Julie Andrews' played in.\\

Did you get "Little Miss Marker twice"?\\
Julie Andrews starred in the 1980 remake of Little Miss Marker and not the original(1934).\\

Title is not a unique field, create a table of IDs   \color{red}in your subquery\color{black}\\

SELECT title, name\\
FROM (movie INNER JOIN casting ON movie.id = casting.movieid) \\
INNER JOIN actor ON actor.id = casting.actorid\\
WHERE ord = 1 \\
AND movieid IN \\
(SELECT movie.id \\
FROM (movie INNER JOIN casting ON movie.id = casting.movieid) \\
INNER JOIN actor ON actor.id = casting.actorid\\
WHERE name = 'Julie Andrews')\\

\subsection{Actors with 15 leading roles}

SELECT name\\
FROM actor INNER JOIN casting ON actor.id = casting.actorid\\
WHERE ord = 1\\
GROUP BY actor.name\\
HAVING SUM(ord)$\geq$ 15\\
ORDER BY name\\

\subsection{}

List the films released in the year 1978 ordered by the number of actors in the cast, then by title.\\

SELECT title\color{red}, COUNT(actor.name)\color{black}\\
FROM (movie INNER JOIN casting ON casting.movieid = movie.id) \\
INNER JOIN actor ON actor.id = casting.actorid\\
WHERE yr='1978'\\
GROUP BY title\\
ORDER BY COUNT(actor.name) DESC, title\\

\color{blue}*Nota. \color{black} En este caso la parte en rojo se puede quitar para que la consulta solo devuelva los nombres de las películas. Sin embargo, lo he dejado así porque era como se daba por correcta la respuesta en la web.

\subsection{}
List all the people who have worked with 'Art Garfunkel'.\\

SELECT name \\
FROM (movie INNER JOIN casting ON movie.id = casting.movieid) \\
INNER JOIN actor ON actor.id = casting.actorid\\
WHERE title IN \color{blue}(SELECT title\\
FROM (movie INNER JOIN casting ON movie.id = casting.movieid) \\
INNER JOIN actor ON actor.id = casting.actorid\\
WHERE name = 'Art Garfunkel') \color{black}AND name != 'Art Garfunkel'\\

\section{Using NULL}

The school includes many departments. Most teachers work exclusively for a single department. Some teachers have no department.

\subsection{NULL, INNER JOIN, LEFT JOIN, RIGHT JOIN}

List the teachers who have NULL for their department.

Why cannot we use =?\\
You might think that the phrase dept=NULL would work here but it doesn't - you have to use the phrase dept IS NULL because NULL is not a value, is one of the three possible final answers in the three-valued logic used for SQL.\\

SELECT name\\
FROM teacher\\
WHERE dept IS NULL\\

\subsection{}

Note the INNER JOIN misses the teachers with no department and the departments with no teacher.\\

SELECT teacher.name, dept.name \\
FROM teacher INNER JOIN dept ON dept.id = teacher.dept \\

\subsection{}

Use a different JOIN so that all teachers are listed.\\

SELECT teacher.name, dept.name\\
FROM teacher LEFT OUTER JOIN dept ON teacher.dept = dept.id\\

\subsection{}

Use a different JOIN so that all teachers are listed.\\

SELECT teacher.name, dept.name\\
FROM teacher LEFT OUTER JOIN dept ON teacher.dept = dept.id\\

\subsection{}

Use a different JOIN so that all departments are listed.\\

SELECT teacher.name, dept.name\\
FROM dept LEFT OUTER JOIN teacher ON dept.id = teacher.dept\\

\subsection{Using the \hyperref{https://sqlzoo.net/wiki/COALESCE}{}{}{COALESCE} function}

Use COALESCE to print the mobile number. Use the number '07986 444 2266' if there is no number given. Show teacher name and mobile number or '07986 444 2266'\\

SELECT name, \color{red}COALESCE(mobile, '07986 444 2266')\color{black}\\
FROM teacher\\

\subsection{}

Use the COALESCE function and a LEFT JOIN to print the teacher name and department name. Use the string 'None' where there is no department.\\

SELECT teacher.name, COALESCE(dept.name, 'None')\\
FROM teacher LEFT OUTER JOIN dept ON teacher.dept = dept.id\\

\subsection{}

Use COUNT to show the number of teachers and the number of mobile phones.\\

SELECT COUNT(teacher.name), COUNT(teacher.mobile)\\
FROM teacher\\

\subsection{}

Use COUNT and GROUP BY dept.name to show each department and the number of staff. Use a RIGHT JOIN to ensure that the Engineering department is listed.\\

SELECT dept.name, COUNT(teacher.name)\\
FROM teacher RIGHT JOIN dept ON teacher.dept = dept.id\\
GROUP BY dept.name\\

\color{blue}*Nota. \color{black}El uso del RIGHT (OUTER) JOIN está desaconsejado (se usa el LEFT (OUTER) JOIN por convenio). Obsérvese que pongo el "OUTER" entre paréntesis, ya que no hay necesidad de escribirlo en la consulta, como se puede observar en esta misma consulta.\\

\subsection{Using \hyperref{https://sqlzoo.net/wiki/CASE}{}{}{CASE}}

Use CASE to show the name of each teacher followed by 'Sci' if the teacher is in dept 1 or 2 and 'Art' otherwise.\\

SELECT teacher.name, CASE \\
WHEN (dept = 1 OR dept = 2) THEN 'Sci'\\
ELSE 'Art'\\
END AS Field\\
FROM teacher\\

\subsection{}

Use CASE to show the name of each teacher followed by 'Sci' if the teacher is in dept 1 or 2, show 'Art' if the teacher's dept is 3 and 'None' otherwise.\\

SELECT teacher.name, CASE \\
WHEN (dept = 1 OR dept = 2) THEN 'Sci'\\
WHEN dept = 3 THEN 'Art'\\
ELSE 'None'\\
END AS Field\\
FROM teacher\\

\section{NSS tutorial (numeric examples)}

National Student Survey 2012\\

The National Student Survey http://www.thestudentsurvey.com/ is presented to thousands of graduating students in UK Higher Education. The survey asks 22 questions, students can respond with STRONGLY DISAGREE, DISAGREE, NEUTRAL, AGREE or STRONGLY AGREE. The values in these columns represent PERCENTAGES of the total students who responded with that answer.\\

The table nss has one row per institution, subject and question (although it has more columns than those, like you can check in the website \hyperref{https://sqlzoo.net/wiki/NSS\_Tutorial}{}{}{https://sqlzoo.net/wiki/NSS\_Tutorial}).\\

\subsection{Check out one row}

The example shows the number who responded for:\\

question 1\\
at 'Edinburgh Napier University'\\
studying '(8) Computer Science'\\
Show the the percentage who STRONGLY AGREE\\

Aquí hay dos posibles opciones, ya que en ningún momento se dice que A\_STRONGLY\_AGREE sea un porcentaje. Si no lo es, la respuesta sería:\\

SELECT (A\_STRONGLY\_AGREE/(SELECT response \\
FROM nss \\
WHERE institution = 'Edinburgh Napier University' AND subject = '(8) Computer Science' AND question =     'Q01')*100) \\
AS Percentage \\
FROM nss\\
WHERE institution = 'Edinburgh Napier University' AND subject = '(8) Computer Science' AND question = \\
'Q01' \\

Pero esta respuesta no es correcta en la web, así que, deberíamos asumir que A\_STRONGLY\_AGREE es un porcentaje, ya que la respuesta que se da por buena es la siguiente:\\

SELECT \color{red}A\_STRONGLY\_AGREE\color{black}\\
FROM nss\\
WHERE question='Q01'\\
AND institution='Edinburgh Napier University'\\
AND subject='(8) Computer Science'\\

\subsection{Calculate how many agree or strongly agree}

Show the institution and subject where the score is at least 100 for question 15.\\



\end{document}

